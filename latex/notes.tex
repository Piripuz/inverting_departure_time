\documentclass{article}

\usepackage[english]{babel}

\usepackage{amsmath}
\usepackage{amssymb}
\usepackage{mathtools}
\usepackage{siunitx}
\usepackage{float}
\usepackage[thinc]{esdiff}
\usepackage{tikz}
\usepackage{pgfplots}
\usepackage{booktabs}
\usepackage{minted}

\usepackage[pdfborderstyle={/S/U/W 0}]{hyperref}

\DeclareMathOperator*{\argmax}{arg\,max}
\DeclareMathOperator*{\argmin}{arg\,min}

\begin{document}

The cost function, in function of departure time \(t_d\), is
\begin{equation}
  \label{eq:cost_td}
  C(t_d) = \alpha tt(t_d) + \beta[t^*-t_d-tt(t_d)]^+ + \gamma[t_d+tt(t_d)-t^*]^+ 
\end{equation}
where
\begin{itemize}
\item \(t^*\) is the desired arrival time
\item \(\alpha\) is the value of time spent travelling
\item \(\beta\) is the value of time spent waiting there
\item \(\gamma\) is the value of time arriving late
\item \(tt(t_d)\) is the time spent travelling if leaving at time \(t_d\)
\item \([x]^+ = \max(0, x)\)
\end{itemize}

It would be helpful to express the cost function in term of the arrival time \(t_a = t_d + tt(t_d)\),
so that the second part of the cost function would greatly simplify
\[ \beta[t^*-t_d-tt(t_d)]^+ + \gamma[t_d+tt(t_d)-t^* = \beta[t^*-t_a]^+ + \gamma[t_a-t^*]^+ \]

But how do we express the first term \(tt(t_d)\) in terms of \(t_a\)?

\begin{equation}
  \label{eq:ta_td}
  t_a(t_d) = t_d + tt(t_d)
\end{equation}

Note that the travel time can be expressed in terms of the arrival time \(t_a\) if and only if the function
\eqref{eq:ta_td} is invertible:
this is because the departure time can be easily reconstructed from the travel time and the arrival time (and viceversa).

Inverting \eqref{eq:ta_td} is not analytically possible a priori, and may not be possible in general.
It only depends on whether the function \(tt(t_d)\) ever grows more than the identity.
For instance, if the travel time is a gausssian
\begin{equation*}
  tt(t_d) = \frac{1}{\sigma\sqrt{2\pi}}\exp\left(-{\frac{x^2}{2\sigma^2}}\right)
\end{equation*}
then the arrival time is invertible if and only if the variance \(\sigma\) satisfies the condition
\[\sigma \leq (2\pi e)^{-\frac{1}{4}} \approx 0.492 \]

In general, assuming that the function grows less than the identity is pretty reasonable,
since in real world I think leaving later results in arriving later.

From now on, assume \eqref{eq:ta_td} is invertible, and \(tt_a(t_a)\) exists.
Moreover, \(\alpha = 1\).

\begin{equation}
  \label{eq:cost_ta}
  C(t_a) = tt_a(t_a) + \beta[t^*-t_a]^+ + \gamma[t_a-t^*]^+ \]
\end{equation}


\end{document}
%%% Local Variables:
%%% mode: LaTeX
%%% TeX-master: t
%%% End:
