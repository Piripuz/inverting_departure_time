\documentclass{article}

\usepackage[english]{babel}

\usepackage{amsmath}
\usepackage{amssymb}
\usepackage{mathtools}
\usepackage{siunitx}
\usepackage{float}
\usepackage[thinc]{esdiff}
\usepackage{tikz}
\usepackage{pgfplots}
\usepackage{booktabs}
\usepackage{minted}

\usepackage[pdfborderstyle={/S/U/W 0}]{hyperref}

\DeclareMathOperator*{\argmax}{arg\,max}
\DeclareMathOperator*{\argmin}{arg\,min}

\begin{document}

The cost function, in function of departure time \(t_d\), is
\begin{equation}
  \label{eq:cost_td}
  C(t_d) = \alpha tt(t_d) + \beta[t^*-t_d-tt(t_d)]^+ + \gamma[t_d+tt(t_d)-t^*]^+ 
\end{equation}
where
\begin{itemize}
\item \(t^*\) is the desired arrival time
\item \(\alpha\) is the value of time spent travelling
\item \(\beta\) is the value of time spent waiting there
\item \(\gamma\) is the value of time arriving late
\item \(tt(t_d)\) is the time spent travelling if leaving at time \(t_d\)
\item \([x]^+ = \max(0, x)\)
\end{itemize}

It would be helpful to express the cost function in term of the arrival time \(t_a = t_d + tt(t_d)\),
so that the second part of the cost function would greatly simplify
\[ \beta[t^*-t_d-tt(t_d)]^+ + \gamma[t_d+tt(t_d)-t^* = \beta[t^*-t_a]^+ + \gamma[t_a-t^*]^+ \]

But how do we express the first term \(tt(t_d)\) in terms of \(t_a\)?

\begin{equation}
  \label{eq:ta_td}
  t_a(t_d) = t_d + tt(t_d)
\end{equation}

Note that the travel time can be expressed in terms of the arrival time \(t_a\) if and only if the function
\eqref{eq:ta_td} is invertible:
this is because the departure time can be easily reconstructed from the travel time and the arrival time (and viceversa).

Inverting \eqref{eq:ta_td} is not analytically possible a priori, and may not be possible in general.
It only depends on whether the function \(tt(t_d)\) ever grows more than the identity.
For instance, if the travel time is a gausssian
\begin{equation*}
  tt(t_d) = \frac{1}{\sigma\sqrt{2\pi}}\exp\left(-{\frac{t_d^2}{2\sigma^2}}\right)
\end{equation*}
then the arrival time is invertible if and only if the variance \(\sigma\) satisfies the condition
\[\sigma \leq (2\pi e)^{-\frac{1}{4}} \approx 0.492 \]

In general, assuming that the function grows less than the identity is pretty reasonable,
since in real world I think leaving later results in arriving later.

From now on, assume \eqref{eq:ta_td} is invertible, and \(tt_a(t_a) = tt(t_d(t_a))\) exists.
Moreover, \(\alpha = 1\).

\begin{equation}
  \label{eq:cost_ta}
  C(t_a) = tt_a(t_a) + \beta[t^*-t_a]^+ + \gamma[t_a-t^*]^+
\end{equation}

The cost function \eqref{eq:cost_ta} could be minimized either at the only non differentiable point (for \(t_a = t^*\))
or at one of the points where its derivative is zero.
\begin{equation}
  \label{eq:cost_diff}
  C'(t_a) =
  \begin{cases}
    tt_a'(t_a) -\beta \quad &\text{if } t_a < t^* \\
    tt_a'(t_a) + \gamma \quad &\text{otherwise}
  \end{cases}
\end{equation}

Setting it equal to zero, the minimum is realized by one of the points

\begin{align*}
  & t_a |\ tt_a'(t_a) = \beta, t_a < t^* \\
  & t_a = t^* \\
  & t_a |\ tt_a'(t_a) = -\gamma, t_a > t^* \\
\end{align*}

Again, let's restrict to the case in which the travel time is gaussian.
I assume that the travel time is gaussian in function of the arrival time as well,
and this can be justified by saying that
\[t_a - t_d >> \max_t | tt'(t)|\]
but \textcolor{red}{I doubt that this is an assumption that can be made}.
Still, assume that
\begin{equation}
  \label{eq:travel_time_gauss}
  tt_a(t_a) = \frac{1}{\sigma\sqrt{2\pi}}\exp\left(-{\frac{t_a^2}{2\sigma^2}}\right)
\end{equation}
and so
\begin{equation}
  \label{eq:travel_time_gauss_diff}
  tt_a'(t_a) = -\frac{t_a}{\sigma^3\sqrt{2\pi}}\exp\left(-{\frac{t_a^2}{2\sigma^2}}\right)
\end{equation}
This can't be inverted analytically and the solutions to the equations above can be found numerically.
(I hope other approaches for minimizing the cost are possible, but nothing comes to my mind right now).

\end{document}
%%% Local Variables:
%%% mode: LaTeX
%%% TeX-master: t
%%% End:
